\documentclass[conference]{IEEEtran}
\IEEEoverridecommandlockouts
% The preceding line is only needed to identify funding in the first footnote. If that is unneeded, please comment it out.
\usepackage{cite}
\usepackage{amsmath,amssymb,amsfonts}
\usepackage{algorithmic}
\usepackage{graphicx}
\usepackage{textcomp}
\usepackage{xcolor}
\usepackage{url}
\usepackage{parskip}
\def\BibTeX{{\rm B\kern-.05em{\sc i\kern-.025em b}\kern-.08em
    T\kern-.1667em\lower.7ex\hbox{E}\kern-.125emX}}
\begin{document}

\title{Estimativa do Índice Glicêmico de Bebidas Comerciais Usando Processamento de
Imagens com Deep Learning para Monitoramento de Saúde\
}

\author{\IEEEauthorblockN{Lorrany B. Amorim Marim\textsuperscript{1}}
\IEEEauthorblockA{\textit{Departamento de Ciência da Computação} \\
\textit{Universidade Federal de Ouro Preto}\\
Ouro Preto, MG, Brasil\\
lorrany.marim@ufop.aluno.edu.br}
\and
\IEEEauthorblockN{André Luiz Carvalho Ottoni\textsuperscript{2}}
\IEEEauthorblockA{\textit{Departamento de Ciência da Computação} \\
\textit{Universidade Federal de Ouro Preto}\\
Ouro Preto, MG, Brasil\\
andre.ottoni@ufop.edu.br}
}

\maketitle

\begin{abstract}
\sloppy
The monitoring and control of blood glucose levels are crucial for individuals with metabolic diseases, such as diabetes
(type 1, type 2, and gestational) and insulin resistance. This research addresses a gap in available tools for glycemic
management, particularly concerning commercial beverages that are widely consumed and can significantly impact glycemic
control. The study proposes an artificial intelligence (AI)-based solution using deep learning techniques to estimate the
glycemic index (GI) of ready-to-drink commercial beverages through image processing. By utilizing convolutional neural
networks (CNNs) and other deep learning architectures, the proposed system aims to classify and estimate the glycemic
impact of various beverages to aid consumers in managing their blood glucose levels effectively. Recent studies have shown
promising results in using deep learning models, including bidirectional recurrent neural networks (RNNs) with attention
mechanisms, for reliable glucose predictions. The developed model is designed to recognize beverage images, estimate GI,
and categorize them as suitable, recommended with caution, or unsuitable for glycemic control, enhancing accessibility and
promoting informed dietary choices for individuals with diabetes and insulin resistance. This study discusses the economic
and health relevance of GI in food and beverages, the challenges of accurately estimating GI in liquid matrices, and the
potential of AI for real-time applications on mobile devices.
\fussy
\end{abstract}


\begin{IEEEkeywords}
\sloppy
Artificial intelligence, glycemic index, diabetes, insulin resistance, blood glucose monitoring, deep learning, convolutional neural networks, image processing, glycemic control, recurrent neural networks, commercial beverages, mobile applications, metabolic diseases.
\fussy
\end{IEEEkeywords}

\section{Introdução}
\sloppy
Este estudo visa abordar o desenvolvimento de ferramen-
tas práticas para indivíduos que precisam monitorar os níveis 
de glicose (açúcar) no sangue, como pessoas com diabetes 
e resistência insulínica (pré-diabéticos). O projeto propõe o 
uso de inteligência artificial para o reconhecimento de bebidas 
comerciais prontas para consumo, vendidas em supermerca--
dos, lojas de conveniência e restaurantes. Através do recon
hecimento por imagem, calcula-se o impacto glicêmico dessas 
bebidas, classificando-as como adequadas, consumíveis com 
recomendações, ou inadequadas.
\par
O controle e monitoramento dos níveis de glicemia no san--
gue são fundamentais para indivíduos com doenças metabóli--
cas, como o diabetes (tipos 1, 2 e gestacional) e a resistência 
insulínica. Essas condições, caracterizadas pela produção ex-
cessiva de glicose (hiperglicemia) ou pela insuficiência na
produção desse açúcar (hipoglicemia), afetam diretamente a
saúde, podendo resultar em complicações graves, como retino--
patia, nefropatia, infarto do miocárdio, neuropatias e até a
morte, se não forem adequadamente controladas\cite{b1}.
\par
Estudos clínicos como o Diabetes Control and Complica--
tions Trial (DCCT) e o United Kingdom Prospective Diabetes
Study (UKPDS) forneceram evidências robustas sobre a im-
portância do controle intensivo da glicemia em pacientes
com diabetes tipo 1 e tipo 2, mostrando uma redução signi
ficativa nas complicações microvasculares e macrovasculares
associadas à doença\cite{b1}. Adicionalmente, o Diabetes Preven-
tion Program (DPP) demonstrou que mudanças no estilo de
vida, combinadas ao controle glicêmico, podem retardar a
progressão da resistência insulínica para o diabetes tipo 2
e reduzir o risco de complicações cardiovasculares\cite{b2}.
\par
Diante do aumento global dos casos de diabetes, que afeta 
8,5\% da população adulta e foi responsável por 1,5 milhão de
mortes em 2019\cite{b3}, é imperativo buscar novas ferramentas 
para auxiliar o controle glicêmico. Uma dessas ferramentas é 
o índice glicêmico (IG), que classifica os alimentos de acordo 
com a rapidez com que elevam os níveis de glicose no sangue. 
Alimentos com baixo IG são absorvidos de forma mais lenta, 
o que ajuda a manter os níveis de glicose estáveis, sendo 
essa uma estratégia essencial para o controle da glicemia em 
pacientes com diabetes e resistência insulínica. Alimentos com 
IG médio produzem uma elevação moderada nos níveis de 
glicose, enquanto alimentos com IG alto são absorvidos ra--
pidamente, o que pode levar a picos de glicose no sangue, 
sendo desaconselhados para pessoas que precisam controlar
rigorosamente a glicemia\cite{b4},\cite{b5}.
\par
A aplicação prática do índice glicêmico (IG), apesar de ser 
amplamente reconhecido como uma ferramenta essencial 
para o controle glicêmico, enfrenta desafios significativos\cite{b6}. 
Esse problema é particularmente notável no caso de bebidas comerciais, 
tanto alcoólicas quanto não alcoólicas, amplamente 
consumidas globalmente. Muitas dessas bebidas apresentam
altos índices glicêmicos devido à sua composição nutricional, 
o que pode levar a elevações rápidas nos níveis de glicose no
sangue, complicando o manejo de condições como o diabetes. 
Além disso, as tabelas nutricionais comumente disponíveis não
incluem o IG dos produtos, o que obriga os consumidores 
a estimarem o impacto glicêmico com base em informações 
limitadas, como o teor de carboidratos e açúcares\cite{b6}.
\par
A escolha do estudo sobre predição de valores glicêmicos 
em bebidas comerciais justifica-se pela relevância econômica e 
social desse setor. Em 2024, o mercado global de bebidas não 
alcoólicas gerou uma receita de aproximadamente US\$1,61
trilhão, enquanto o setor de bebidas alcoólicas alcançou cerca 
de US\$1,667 bilhões\cite{b7},\cite{b8}. Esses valores refletem a alta 
demanda por bebidas comerciais e a necessidade de ferra-
mentas precisas para auxiliar o consumidor na compreensão 
dos impactos glicêmicos desses produtos, especialmente para 
aqueles que necessitam monitorar a glicemia regularmente.
\par
Samir et al.\cite{b44} propoe dispositivos para monitoramento de IG de alimentos por imagem e atividades fisica, em outro estudo Afnan Ahmed Crystal et al.\cite{b45} discute a integração de sistemas de reconhecimento de alimentos por imagem (FIRS) com redes neurais convulucionais 

Para enfrentar essa lacuna, propomos uma solução baseada 
em inteligência artificial (IA), utilizando redes neurais artifi-
ciais e aprendizado profundo \cite{b16}. Estudos recentes indicam 
que a aplicação de IA na predição de índices glicêmicos tem 
mostrado resultados promissores, com o uso de modelos de 
classificação de imagens alimentares para identificar categorias 
de índice glicêmico (IG) de forma automatizada \cite{b9}. Essa 
tecnologia, ao reconhecer e classificar imagens capturadas por 
dispositivos móveis, como smartphones, em tempo real, pode-
ria consultar uma base de dados previamente treinada para 
estimar o IG das bebidas analisadas \cite{b17}. O sistema proposto 
forneceria, assim, uma ferramenta prática para os consumi-
dores classificarem a adequação de uma bebida para consumo, 
com base na estimativa do IG, contribuindo para o controle 
glicêmico e a prevenção de complicações associadas ao dia-
betes, por meio de uma análise de IG prática e acessível para 
bebidas comerciais.
\fussy

\section{Referencial Teórico}
\subsection{Controle Glicêmico e Impacto na Saúde de Pacientes \\Diabéticos e Resistência Insulínica}
\sloppy
A insulina, hormônio essencial produzido pelo pâncreas, 
é responsável por facilitar a entrada de glicose nas células musculares, 
hepáticas e adiposas, onde é utilizada como fonte de 
energia. Esse suprimento de glicose é proveniente da ali-mentação 
e também sintetizado pelo fígado em situações de 
demanda, como o jejum. Após as refeições, o aumento dos 
níveis de glicose no sangue estimula a liberação de insulina, 
regulando a glicemia e mantendo-a dentro de níveis saudáveis
\cite{b20}. 
\par
A resistência insulínica, caracterizada pela resposta inefi-
ciente das células à insulina, é um importante fator de risco
para o desenvolvimento do diabetes tipo 2. Segundo o Nati-
onal Institute of Diabetes and Digestive and Kidney Diseases
(NIDDK), cerca de 50\% dos indivíduos com pré-diabetes, 
que frequentemente apresentam resistência insulínica, podem 
evoluir para o diabetes tipo 2 se não houver intervenção\cite{b20}. A 
resistência prolongada pode levar ao esgotamento das células
beta pancreáticas, reduzindo a produção de insulina e agra--
vando a hiperglicemia\cite{b21}. Já o diabetes tipo 1 é marcado
pela destruição autoimune das células beta, resultando em uma 
produção mínima ou nula de insulina e em hiperglicemia per--
sistente, caracterizando essa forma da doença\cite{b22}.
\par
O controle glicêmico é fundamental tanto para o tratamento 
do diabetes quanto para a prevenção de complicações como 
doenças cardiovasculares e neuropatias\cite{b1},\cite{b12}. Em pacientes 
com resistência à insulina, mudanças no estilo de vida —
incluindo dieta  com baixo índice glicêmico e aumento da ativi-
dade física — são recomendadas para minimizar a progressão 
para o diabetes tipo 2\cite{b11},\cite{b5}. Para diabéticos, o controle 
glicêmico adequado reduz o risco de complicações micro e
macrovasculares, beneficiando tanto pacientes com diabetes tipo 1 quanto 
tipo 2\cite{b1},\cite{b12},\cite{b13}.
\fussy

\subsection{Pesquisas e Estudos do Impacto na Saúde Através do \\Controle Glicêmico}
\sloppy
Em 2024, o Songyuan Community Health Center, na China, 
acompanhou 359 pacientes com diabetes tipo 2 por seis meses,
monitorando o controle glicêmico através de registros e ex-
ames periódicos. O estudo demonstrou que 62,7\% dos pa-
cientes alcançaram índices glicêmicos mais estáveis, atribuídos 
a intervenções alimentares e ao ajuste de medicamentos, com 
efeitos positivos nos resultados clínicos\cite{b10}. 
\par
Estudos históricos, como o Diabetes Control and Complications Trial (DCCT) e 
o Epidemiology of Diabetes Interventions and Complications 
(EDIC), liderados pelo NIDDK, mostraram que o controle 
intensivo da glicemia reduz significativamente as complicações 
microvasculares (retinopatia, nefropatia e neuropatia). A per--
sistência dos efeitos do controle intensivo, mesmo após a norma-
lização dos níveis de HbA1c, estabeleceu o conceito de "memó-
ria metabólica". Além disso, o EDIC observou uma redução 
de 58\% em eventos cardiovasculares graves entre os pacientes 
com diabetes tipo 1 submetidos à terapia intensiva\cite{b12}. 
\par
De forma semelhante, o UK Prospective Diabetes Study (UKPDS) 
avaliou o controle glicêmico em pacientes com diabetes tipo 
2, demonstrando que a redução nos níveis de HbA1c diminuiu 
em 25\% as complicações microvasculares, embora os efeitos 
sobre complicações cardiovasculares fossem menos expres-
sivos. No entanto, o conceito de "memória metabólica" obser-
vado no DCCT foi reforçado no UKPDS, sugerindo benefícios 
cardiovasculares com o controle prolongado\cite{b14}. 
\par
Outros estudos, como o Action to Control Cardiovascular Risk in Dia-
betes (ACCORD), apoiado por instituições americanas como 
o National Heart, Lung, and Blood Institute (NHLBI) e o 
Centers for Disease Control and Prevention (CDC), exami-
naram a eficácia do controle intensivo da glicemia em pa-
cientes com diabetes tipo 2 de alto risco cardiovascular. O 
estudo concluiu que, para esse perfil, o controle padrão de 
HbA1c entre 7,0\% e 7,9\% foi mais seguro que um controle 
intensivo, sugerindo que o monitoramento contínuo pode ser 
uma estratégia mais apropriada para reduzir a mortalidade\cite{b18}. 
\par
Por fim, o Veterans Affairs Diabetes Trial (VADT) in-
vestigou a relação entre aterosclerose e controle intensivo da 
glicemia em pacientes com diabetes tipo 2. A pesquisa reve-
lou que o controle intensivo foi mais eficaz na redução de 
eventos cardiovasculares entre pacientes com baixo índice de 
cálcio coronariano, enquanto os pacientes com aterosclerose 
avançada mostraram menor benefício, destacando a necessi-
dade de individualização do tratamento\cite{b19}. 
\par
Esses estudos, juntamente com o Finnish Diabetes Prevention Study (DPS), 
que observou uma redução de 58\% no risco de
diabetes tipo 2 com intervenções no estilo de vida, ressaltam a importância de estratégias preventivas e de controle
intensivo baseadas em características específicas dos pacientes\cite{b23}.
\par
Portanto, estudos sugerem que dietas de baixo índice glicêmico (IG) podem reduzir significativamente os riscos associados
ao diabetes e a doenças cardiovasculares, conforme discutido anteriormente. As evidências indicam que essas dietas melhoram
o controle glicêmico em comparação com dietas de alto IG, proporcionando benefícios como a redução dos níveis de HbA1c e
proteínas glicadas, o que resulta em uma melhora expressiva na glicemia média em pacientes com diabetes tipo 1 e tipo 2.
Além disso, uma dieta de baixo IG pode ter efeitos clínicos semelhantes aos de alguns agentes farmacológicos destinados ao
controle da hiperglicemia pós-prandial, posicionando-se como uma estratégia eficaz para a gestão do diabetes \cite{b37}.
\par
Em um estudo de Jenkins et al. sobre os efeitos metabólicos de uma dieta de baixo IG, foi observado que o consumo de
alimentos e bebidas de baixo IG resultou em uma redução significativa na glicose ao longo de 12 horas, em comparação com
dietas de alto IG. Além disso, houve uma diminuição na proteína sérica glicada (frutosamina), indicando uma redução nos
níveis médios de glicose no sangue nas últimas 1 a 3 semanas, o que sugere uma melhora no controle glicêmico. Observou-se
também uma redução de 32\% na secreção de insulina e uma diminuição de 15\% no colesterol total. \cite{b38}.
\par
Outro estudo realizado por Brand-Miller e Buyken (2012) apontou para a redução dos riscos cardiovasculares associados a
dietas de baixo IG, em comparação com dietas de alto IG. Esses achados indicam uma melhora no controle de peso em pessoas
com sobrepeso, condição frequentemente associada à resistência insulínica, auxiliando na prevenção da recuperação de peso
após a perda inicial. Dietas de baixo IG também foram associadas à melhora em marcadores inflamatórios e à redução de
doenças inflamatórias não cardiovasculares e não oncológicas. \cite{b39}.
\par
Para aproveitar os benefícios das dietas de baixo IG, indivíduos que
necessitam de controle glicêmico devido a condições de
saúde devem considerar tanto alimentos sólidos quanto líquidos. Embora a
maioria dos estudos e orientações sobre IG
concentre-se em alimentos sólidos, é essencial conhecer o IG das bebidas
comercializadas, uma vez que elas representam uma
parte significativa do consumo total diário de carboidratos.
\fussy

\subsection{Importância do Índice Glicêmico (IG) para o Controle \\Glicêmico}
\sloppy
O índice glicêmico (IG) é uma medida que classifica os 
alimentos contendo carboidratos de acordo com seu impacto 
na glicemia pós-prandial, comparando a resposta glicêmica 
com uma referência, geralmente glicose pura ou pão branco 
\cite{b4}, \cite{b27}. Os carboidratos são o principal macronutriente 
influente no IG devido à sua absorção rápida e direta, que 
eleva os níveis de glicose no sangue. Alimentos de alto IG 
causam aumentos rápidos na glicemia, enquanto alimentos 
de baixo IG promovem elevações mais lentas e moderadas 
\cite{b24},\cite{b5}. 
\par
Essa classificação é especialmente relevante para 
indivíduos com diabetes e resistência insulínica, nos quais a 
ingestão de carboidratos de alto IG pode exacerbar o descon-
trole glicêmico, exigindo ajustes na administração de insulina 
e dificultando o controle da doença \cite{b25}, \cite{b26}. Para esses 
indivíduos, o consumo de alimentos de baixo IG pode ajudar 
a reduzir variações glicêmicas, melhorando a sensibilidade 
à insulina e contribuindo para um manejo dietético mais
seguro. 
\par
Os alimentos são classificados segundo seus valores 
de índice glicêmico (IG) em três categorias: baixo \((\le55)\), médio \((56\)-
\(69)\) e alto \((\ge 70)\) \cite{b4}, \cite{b27}. Esses valores são estabelecidos 
pela medição da área sob a curva glicêmica (AUC) nas duas 
horas subsequentes à ingestão de uma porção padrão de 50
gramas de carboidratos disponíveis. Esse método fornece uma 
base para identificar alimentos que causam menor impacto nos 
níveis de glicose. A fórmula do IG foi originalmente proposta 
por Jenkins et al., que calcularam o IG como a razão entre a
AUC do alimento teste e a AUC de um alimento de referência 
(geralmente glicose pura), multiplicada por 100, conforme a 
seguinte fórmula:

\begin{equation*}
IG = \left(\frac{\text{AUC do alimento teste}}{\text{AUC do alimento referência}}\right)\times100 
\end{equation*} como mostrado em \cite{b4}.
\par
Essa abordagem fornece uma referência para identificar ali-
mentos com menor impacto glicêmico, essencial para o con
trole dietético em condições de controle glicêmico como o 
diabetes \cite{b4}. 
\par
Diversos fatores podem influenciar o IG, como 
a composição do alimento (tipo de carboidrato, presença de 
fibras e proteínas), o método de preparo e a estrutura física, 
sendo que alimentos processados tendem a ter IG mais alto 
\cite{b28}. 
\par
Segundo Jenkins e Brand-Miller, o uso do  IG como ferra-
menta no controle glicêmico é uma estratégia eficaz e prática, 
especialmente para pacientes com diabetes e  resistência in--
sulínica, pois permite a escolha de alimentos que minimizam 
as flutuações glicêmicas, promovendo um controle metabólico 
mais estável \cite{b4}, \cite{b27}, \cite{b28}.
\fussy

\subsection{Definição nos Desafios na Predição do IG em Bebidas \\Comerciais}
\sloppy
O mercado global de bebidas alcoólicas e não alcoólicas 
ocupa uma fatia significativa do consumo, evidenciando-se
pela estimativa de um consumo médio per capita de 28,22 
litros para bebidas alcoólicas e 105,70 litros para bebidas não
alcoólicas em 2024, com uma receita combinada de US\$3,77 trilhões 
\cite{b7}, \cite{b8}. Apesar de seu impacto econômico e presença 
difundida, há limitações quanto à acessibilidade de dados nu
tricionais detalhados, especialmente informações sobre o índice 
glicêmico (IG). O consumidor final frequentemente subestima 
a quantidade de carboidratos presentes em bebidas, 
associando-os mais a alimentos sólidos, como pães e massas, do que 
a bebidas como refrigerantes, sucos, bebidas energéticas e 
alcoólicas fermentadas (cervejas e saquês) \cite{b6}. Essas limitações 
nutricionais dificultam o controle glicêmico, particularmente 
para indivíduos com necessidades específicas, como diabéticos
e pessoas com resistência insulínica, que precisam monitorar o 
impacto glicêmico dos produtos consumidos. A complexidade 
na análise do IG em bebidas é reforçada por estudos que 
demonstram a variabilidade na composição de carboidratos 
em líquidos, o que exige metodologias avançadas, como a cro-
matografia líquida de alta performance (HPLC) e a eletroforese 
capilar, para identificar e quantificar precisamente os tipos 
de açúcares presentes \cite{b29}. Embora tais técnicas permitam a 
detecção precisa de carboidratos, elas não são viáveis para uso 
em larga escala ou acessíveis ao consumidor final. Além disso, 
métodos baseados na composição de macronutrientes, con-
forme descrito em estudos recentes, sugerem que a presença
de proteínas e gorduras pode modificar o impacto glicêmico 
dos carboidratos, criando desafios adicionais na predição do 
IG \cite{b30}. Portanto, é fundamental desenvolver ferramentas que 
possam fornecer estimativas rápidas e precisas do IG de be--
bidas comerciais, permitindo que consumidores façam esco--
lhas informadas para o controle glicêmico. A análise do índice 
glicêmico (IG) em matrizes líquidas apresenta uma complexi
dade maior do que em alimentos sólidos devido à variabi--
lidade na composição e concentração dos carboidratos pre--
sentes nas bebidas. Estudos indicam que, em bebidas, a taxa 
de absorção de açúcares pode ser influenciada por diversos 
fatores, como a presença de outros componentes macronutri-
entes, a acidez e a viscosidade do líquido. Essa complexi-
dade demanda o uso de técnicas avançadas de análise, como 
cromatografia líquida de alta performance (HPLC) e eletro-
forese capilar, as quais são capazes de identificar e quan--
tificar açúcares individuais, oferecendo uma análise precisa 
das concentrações de glicose, frutose e sacarose em difere-
ntes tipos de bebidas \cite{b40}, \cite{b41}. Tais metodologias, embora 
precisas, enfrentam limitações práticas: são caras, requerem 
equipamentos especializados e habilidades laboratoriais, o que 
inviabiliza sua aplicação em larga escala e acessibilidade direta 
ao consumidor \cite{b42}. Dessa forma, o desenvolvimento de novas
ferramentas que permitam uma estimativa confiável e rápida
do IG em bebidas comerciais é crucial para que indivíduos 
com necessidades específicas de controle glicêmico possam 
tomar decisões informadas sobre seu consumo. Além disso, a 
disponibilização de dados mais acessíveis e precisos sobre o 
IG de bebidas poderia ter implicações positivas para a saúde 
pública, ao promover escolhas alimentares mais saudáveis e 
apoiar iniciativas voltadas à prevenção de doenças crônicas, 
como diabetes e obesidade \cite{b43}.
\fussy

\subsection{Avanços no Uso de Deep Learning para Estimativos de IG}
\sloppy
Os avanços destacados no uso de deep learning para predição 
de glicemia neste artigo incluem a utilização de redes neurais
recorrentes (RNNs) bidirecionais com mecanismo de atenção e 
aprendizagem evidencial para uma predição de glicose perso--
nalizada e precisa. A combinação de RNNs com meta-aprendi--
zagem adaptativa e aprendizado evidencial é particularmente
significativa, pois permite gerar previsões confiáveis e adaptar 
rapidamente os modelos a novos dados com precisão melho-
rada \cite{b34}, \cite{b35}. Além disso, o artigo introduz a "Fast-adaptive 
and Confident Neural Network" (FCNN), que, ao utilizar re-
des RNN com meta aprendizagem agnóstica, reduz a neces-
sidade de grandes volumes de dados iniciais para adaptar o 
modelo a pacientes individuais \cite{b36}. Esse avanço é crucial 
para aplicações móveis, como aplicativos de monitoramento 
glicêmico em tempo real. A integração de camadas de atenção 
e uma camada de saída evidencial melhora a capacidade do 
modelo de focar em dados críticos e de atribuir um nível de 
confiança a cada previsão, ajudando na mitigação de eventos 
adversos em contexto clínico \cite{b33}, \cite{b34}.
\fussy

\subsection{Frameworks e Técnicas de Processamento de Imagem \\para Predição Glicêmica}
\sloppy
O uso de deep learning tem revolucionado a análise e predi--
ção do índice glicêmico (IG) em alimentos, com abordagens
inovadoras baseadas em visão computacional e redes neurais 
convolucionais (CNNs). Estudos recentes, como o de Khan 
et al., exploram CNNs para identificar e classificar alimen-
tos, alcançando até 98,22\% de acurácia em frutas e 84,30\%
em alimentos variados \cite{b31}. Esses modelos, que dispensam 
pré-processamento extensivo, são altamente escaláveis, faci--
litando sua integração em aplicativos móveis para monitora-
mento de glicemia e consumo de carboidratos. Pesquisas como  
as de Muresan e Oltean aprimoraram o uso de imagens RGB, 
grayscale e HSV, alcançando 97,04\% de acurácia na classifica--
ção de frutas, um avanço importante para aplicações práticas 
de estimativa de IG \cite{b32}. Modelos como LSTM e redes com 
atenção aplicadas à predição de glicose demonstram a robustez 
das arquiteturas de deep learning ao capturar padrões tempo--
rais e fornecer previsões confiáveis \cite{b33}. Além disso, o uso 
de redes com meta-aprendizagem e evidências deep learning 
permite adaptação rápida a novos usuários e maior precisão, 
ampliando as possibilidades de personalização e aplicação em 
tempo real para controle glicêmico \cite{b34}.
\fussy



\begin{thebibliography}{00}
\bibitem{b1} S. V. Edelman, "Importance of glucose control", \textit{Medical Clinics of North America}, vol. 82, no. 4, pp. 665--687, 1998.
\bibitem{b2} I. Conget and M. Giménez, "Glucose control and cardiovascular disease—Is it important? No", \textit{Diabetes Care},
vol. 32, suppl. 2, p. S335, 2009.
\bibitem{b3} World Health Organization, "Diabetes", \textit{World Health Organization}, 2023. [Online]. Available: \url{https://www.who.int/news-room/fact-sheets/detail/diabetes}. [Accessed: Oct. 21, 2024].
\bibitem{b4} D. J. A. Jenkins and T. M. S. Wolever and R. H. Taylor and H. Barker and H. Fielden and J. M. Baldwin and D. V. Goff,
"Glycemic index of foods: A physiological basis for carbohydrate exchange", \textit{The American Journal of Clinical Nutrition},
vol. 34, no. 3, pp. 362–366, 1981.
\bibitem{b5} J. C. Brand-Miller, "Importance of glycemic index in diabetes", \textit{The American Journal of Clinical Nutrition},
vol. 59, no. 3, pp. 747S–752S, 1994. DOI: 10.1093/ajcn/59.3.747S.
\bibitem{b6} G. Cowburn and L. Stockley, "Consumer understanding and use of nutrition labelling: A systematic review",
\textit{Public Health Nutrition}, vol. 8, no. 1, pp. 21–28, 2004. DOI: 10.1079/PHN2005666.
\bibitem{b7} Statista, "Alcoholic drinks – worldwide", \textit{Statista}, 2024. [Online]. Available: \url{https://www.statista.com/outlook/cmo/alcoholic-drinks/worldwide#revenue}. [Accessed: Oct. 25, 2024].
\bibitem{b8} Statista, "Non-alcoholic drinks – worldwide", \textit{Statista}, 2024. [Online]. Available: \url{https://www.statista.com/outlook/cmo/non-alcoholic-drinks/worldwide}. [Accessed: Oct. 25, 2024].
\bibitem{b9} M. I. Khan, B. Acharya, and R. K. Chaurasiya, "Automatic Prediction of Glycemic Index Category from Food
Images Using Machine Learning Approaches", \textit{Arabian Journal for Science and Engineering}, vol. 47, no. 5, pp. 10823–10846,
2022. DOI: 10.1007/s13369-022-06754-0.
\bibitem{b10} L. Fang, Q. Li, and J. Ning, "Assessment of community-managed blood glucose control in patients with diabetes
mellitus in Shenzhen, China", \textit{Journal of Diabetes}, vol. 16, no. e13573, 2024. DOI: 10.1111/1753-0407.13573
\bibitem{b11} J. C. Seidell, "Obesity, insulin resistance and diabetes—a worldwide epidemic", \textit{British Journal of Nutrition},
vol. 83, no. S1, pp. S5–S8, 2000, DOI: 10.1017/S000711450000088X.
\bibitem{b12}  D. M. Nathan, for the DCCT/EDIC Research Group, "The Diabetes Control and Complications Trial/Epidemiology
of Diabetes Interventions and Complications Study at 30 years: Overview", \textit{Diabetes Care}, vol. 37, no. 1, pp. 9–16, 2014,
DOI: 10.2337/dc13-2112.
\bibitem{b13} DCCT Research Group, "Diabetes Control and Complications Trial (DCCT) update", \textit{Diabetes Care}, vol. 13, no. 4,
pp. 427–433, 1990, DOI: 10.2337/diacare.13.4.427.
\bibitem{b14} P. King, I. Peacock, and R. Donnelly, "The UK Prospective Diabetes Study (UKPDS): Clinical and therapeutic
implications for type 2 diabetes", \textit{British Journal of Clinical Pharmacology}, vol. 48, no. 5, pp. 643–648, 1999, DOI:
10.1046/j.1365-2125.1999.00092.x.
\bibitem{b15} The Diabetes Prevention Program Research Group, "The Diabetes Prevention Program (DPP): Description of
lifestyle intervention", \textit{Diabetes Care}, vol. 25, no. 12, pp. 2165–2171, 2002, DOI: 10.2337/diacare.25.12.2165.
\bibitem{b16} J. Mantas, A. Hasman, and M. S. Househ, Eds., "Artificial Intelligence in Medicine: Knowledge Representation
and Transparent Models", \textit{Studies in Health Technology and Informatics}, vol. 287, Springer, 2021, DOI: 10.1007/978-3-030
89011-7.
\bibitem{b17} F. Sultana, A. Sufian, and P. Dutta, "Advancements in image classification using convolutional neural
network", \textit{Proceedings of the 2018 Fourth International Conference on Research in Computational Intelligence and
Communication Networks (ICRCICN)}, Kolkata, India, 2018, pp. 122–129, DOI: 10.1109/ICRCICN.2018.8718718.
\bibitem{b18} The ACCORD Study Group, "Action to Control Cardiovascular Risk in Diabetes (ACCORD) trial: Design and
methods”, \textit{The American Journal of Cardiology}, vol. 99, no. 12A, pp. 21i–33i, 2007, DOI: 10.1016.
\bibitem{b19} P. D. Reaven et al., "Intensive glucose-lowering therapy reduces cardiovascular disease events in Veterans
Affairs Diabetes Trial participants with lower calcified coronary atherosclerosis", \textit{Diabetes}, vol. 58, no. 11, pp. 2642
2648, 2009, DOI: 10.2337/db09-0618.
\bibitem{b20} NIDDK, “Prediabetes and Insulin Resistance", \textit{National Institute of Diabetes and Digestive and Kidney Diseases}. [Online]. Available:\url{https://www.niddk.nih.gov/health-information/diabetes/overview/what-is-diabetes/prediabetes-insulin-resistance#causes}. [Accessed Oct. 30, 2024].
\bibitem{b21} IDF, "About Type 2 Diabetes", International Diabetes Federation. [Online]. Available: https://idf.org/about
diabetes/type-2-diabetes/. [Accessed Oct. 30, 2024].
\bibitem{b22} IDF, "About Type 1 Diabetes", International Diabetes Federation. [Online]. Available: https://idf.org/about
diabetes/type-1-diabetes/. [Accessed Oct. 30, 2024].
\bibitem{b23} J. Lindström et al., "The Finnish Diabetes Prevention Study (DPS): Lifestyle intervention and 3-year results
on diet and physical activity", Diabetes Care, vol. 26, no. 12, pp. 3230–3236, Dec. 2003, DOI: 10.2337/diacare.26.12.3230.
\bibitem{b24} E. B. Hamaker and A. A. Tuncil, "Slowly digestible starch: concept, mechanism, and proposed extended glycemic
index," *Critical Reviews in Food Science and Nutrition*, vol. 54, no. 9, pp. 1235–1247, 2014, DOI:
10.1080/10408390903372466.
\bibitem{b25} P. D. McArdle, D. Mellor, S. Rilstone, and J. Taplin, "The role of carbohydrate in diabetes management,"
Practical Diabetes, vol. 33, no. 7, pp. 237–242, 2016.
\bibitem{b26} D. H. Bessesen, "The Role of Carbohydrates in Insulin Resistance," Journal of Nutrition, vol. 131, no. 10,
pp. 2782S–2786S, 2001.
\bibitem{b27} T. M. S. Wolever and D. J. A. Jenkins, The Glycemic Index: A Physiological Classification of Dietary
Carbohydrate, Oxford University Press, 2002.
\bibitem{b28} F. S. Atkinson, K. Foster-Powell, and J. C. Brand-Miller, "International tables of glycemic index and
glycemic load values: 2008," *Diabetes Care*, vol. 31, no. 12, pp. 2281–2283, 2008.
\bibitem{b29} S. A. Ross, K. L. Lamkin, and B. W. Vander Jagt, "Carbohydrate analysis of foods and beverages by capillary
electrophoresis and high-performance liquid chromatography," Journal of Chromatography B, vol. 792, no. 1, pp. 219–227,
2003. DOI: 10.1016/S1570-0232(03)00377-5.
\bibitem{b30} A. Rytz, D. Adeline, K.-A. Lê, D. Tan, L. Lamothe, O. Roger, and K. Macé, "Predicting Glycemic Index and
Glycemic Load from Macronutrients to Accelerate Development of Foods and Beverages with Lower Glucose Responses,"
Nutrients, vol. 11, no. 5, p. 1172, 2019, DOI: 10.3390/nu11051172.
\bibitem{b31} M. I. Khan, B. Acharya, and R. K. Chaurasiya, "Automatic Prediction of Glycemic Index Category from Food
Images Using Machine Learning Approaches," Arabian Journal for Science and Engineering, vol. 47, no. 5, pp. 10823–10846,
2022. DOI: 10.1007/s13369-022-06754-0.
\bibitem{b32} F. L. Pereira, M. T. F. Andrade, e P. C. Cortez, "Classificação do Índice Glicêmico a partir de Imagens de
Alimentos com Redes Neurais Convolucionais," Anais do XXIV Encontro Nacional de Inteligência Artificial e Computacional
(ENIAC), pp. 1-10, 2020.
\bibitem{b33} F. Zhu, Y. Li, Y. Shen, Z. Shen, and J. Li, "Deep learning for blood glucose level prediction: How well do
models generalize across different data sets?" PLoS ONE, vol. 16, no. 2, pp. e0245238, 2021. DOI:
10.1371/journal.pone.0245238.
\bibitem{b34} T. Zhu, K. Li, P. Herrero, e P. Georgiou, "Personalized Blood Glucose Prediction for Type 1 Diabetes Using
Evidential Deep Learning and Meta-Learning," IEEE Transactions on Biomedical Engineering, vol. 70, no. 1, pp. 193-204, Jan.
2023. DOI: 10.1109/TBME.2022.3187703.
\bibitem{b35} S. Langarica et al., "Deep Learning-Based Glucose Prediction Models: A Guide for Practitioners and a Curated
Dataset for Improved Diabetes Management," IEEE Open Journal of Engineering in Medicine and Biology, vol. 5, pp. 467–475,
2024, doi: 10.1109/OJEMB.2024.3365290.
\bibitem{b36} T. Zhu, K. Li, P. Herrero, e P. Georgiou, "Personalized Blood Glucose Prediction for Type 1 Diabetes Using
Evidential Deep Learning and Meta-Learning," IEEE Transactions on Biomedical Engineering, vol. 70, no. 1, pp. 193-204, Jan.
2023, doi: 10.1109/TBME.2022.3187703.
\bibitem{b37} J. Brand-Miller, S. Hayne, P. Petocz, e S. Colagiuri, "Low–Glycemic Index Diets in the Management of
Diabetes: A meta-analysis of randomized controlled trials," Diabetes Care, vol. 26, no. 8, pp. 2261–2267, 2003.
\bibitem{b38} D. J. A. Jenkins, T. M. S. Wolever, G. R. Collier, A. Ocana, A. V. Rao, G. Buckley, Y. Lam, A. Mayer, e L. U.
Thompson, "Metabolic effects of a low-glycemic-index diet," American Journal of Clinical Nutrition, vol. 46, no. 6, pp. 968
975, 1987.
\bibitem{b39} J. Brand-Miller e A. E. Buyken, "The glycemic index issue," Current Opinion in Lipidology, vol. 23, no. 1,
pp. 62-67, 2012.
\bibitem{b40} S. Wolever, "The Glycemic Index in Practice," American Journal of Clinical Nutrition, vol. 89, no. 2, pp. 418
420, 2019.
\bibitem{b41} J. D. Brennan e A. D. Marchini, "Capillary Electrophoresis for Sugars in Beverage Analysis," Journal of
Analytical Methods in Chemistry, vol. 32, no. 4, pp. 56–62, 2020.
\bibitem{b42} P. P. Rubino, "Challenges and Opportunities in Glycemic Index Testing," Food Chemistry, vol. 285, pp. 232
240, 2021.
\bibitem{b43} R. T. Thompson e M. J. Smith, "Public Health Impacts of Glycemic Index Awareness," Journal of Public Health
Nutrition, vol. 27, no. 6, pp. 1090–1098, 2021.
\end{thebibliography}


\end{document}

%artigo: \bibitem{b1} J. Smith and A. Brown, "Deep learning for glycemic index prediction," \textit{IEEE Transactions on Neural Networks}, vol. 29, no. 3, pp. 123--134, 2023. 

%site: \bibitem{b2} World Health Organization, "Diabetes facts and statistics," \textit{WHO Official Website}. [Online]. Available: https://www.who.int/diabetes. [Acessado: 26/11/2024].

%livro: \bibitem{b3} J. Doe and M. Johnson, \textit{Artificial Intelligence in Health Applications}, 2\textsuperscript{a} ed. New York, NY, USA: Springer, 2021.

%conferencia: \bibitem{b4} R. Lee, K. Kim, and T. Zhang, "Advances in glycemic index prediction using AI," em \textit{Proc. IEEE Int. Conf. Artificial Intelligence}, London, UK, 2023, pp. 45--50.


